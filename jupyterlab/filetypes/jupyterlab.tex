\documentclass[12pt]{article}

\usepackage{amsmath}    % need for subequations
\usepackage{graphicx}   % need for figures
\usepackage{verbatim}   % useful for program listings
\usepackage{color}      % use if color is used in text
\usepackage{subfigure}  % use for side-by-side figures
\usepackage{hyperref}   % use for hypertext links, including those to external documents and URLs

% don't need the following. simply use defaults
\setlength{\baselineskip}{16.0pt}    % 16 pt usual spacing between lines

\setlength{\parskip}{3pt plus 2pt}
\setlength{\parindent}{20pt}
\setlength{\oddsidemargin}{0.5cm}
\setlength{\evensidemargin}{0.5cm}
\setlength{\marginparsep}{0.75cm}
\setlength{\marginparwidth}{2.5cm}
\setlength{\marginparpush}{1.0cm}
\setlength{\textwidth}{150mm}

\begin{comment}
\pagestyle{empty} % use if page numbers not wanted
\end{comment}

% above is the preamble

\begin{document}

\begin{center}
{\large Introduction to \LaTeX} \\ % \\ = new line
\copyright 2006 by Harvey Gould \\
December 5, 2006 123
\end{center}

\section{Introduction}
\TeX\ looks more difficult than it is. It is
almost as easy as $\pi$. See how easy it is to make special
symbols such as $\alpha$,
$\beta$, $\gamma$,
$\delta$, $\sin x$, $\hbar$, $\lambda$, $\ldots$ We also can make
subscripts
$A_{x}$, $A_{xy}$ and superscripts, $e^x$, $e^{x^2}$, and
$e^{a^b}$. We will use \LaTeX, which is based on \TeX\ and has
many higher-level commands (macros) for formatting, making
tables, etc. More information can be found in latex.

We just made a new paragraph. Extra lines and spaces make no
difference. Note that all formulas are enclosed by
\$ and occur in \textit{math mode}.

The default font is Computer Modern. It includes \textit{italics},
\textbf{boldface},
\textsl{slanted}, and \texttt{monospaced} fonts.

\section{Equations}
Let us see how easy it is to write equations.
\begin{equation}
\Delta =\sum_{i=1}^N w_i (x_i - \bar{x})^2 .
\end{equation}
It is a good idea to number equations, but we can have a
equation without a number by writing
\begin{equation}
P(x) = \frac{x - a}{b - a} , \nonumber
\end{equation}
and
\begin{equation}
g = \frac{1}{2} \sqrt{2\pi} . \nonumber
\end{equation}

We can give an equation a label so that we can refer to it later.
\begin{equation}
\label{eq:ising}
E = -J \sum_{i=1}^N s_i s_{i+1} ,
\end{equation}
Equation expresses the energy of a configuration
of spins in the Ising model.\footnote{It is necessary to process (typeset) a
file twice to get the counters correct.}

Examples of more complicated equations:
\begin{equation}
I = \! \int_{-\infty}^\infty f(x)\,dx \label{eq:fine}.
\end{equation}
We can do some fine tuning by adding small amounts of horizontal
spacing:
\begin{verbatim}
 \, small space       \! negative space
\end{verbatim}
as is done in Eq. 111


{\small \noindent Updated 5 December 2006.}
\end{document}
